\documentclass[red]{beamer} %pode mudar a cor do beamer, mas so para cores padrao
\usepackage[brazil]{babel}
\usepackage[T1]{fontenc}
\usepackage{amsthm,amsfonts}
\usepackage{graphicx}
\usepackage{graphicx,color}
\usepackage[active]{srcltx}
\usepackage[latin1]{inputenc}
\usepackage[all]{xy}
\usepackage{mathrsfs}
\usepackage{listings}
\usepackage{url}


%%%%%%%%%%%%%%%%%%%%%%%%%%%%%%%%%%%%%%%%%%%%%%%%%%%%%%%%%%%%%%%%%%%%%%%%%%%%%%%%%%%%%%%%%%%%%%%%%%%%%%%%%
%.................................. Defini\c{c}\~{a}o de comandos........................................
%%%%%%%%%%%%%%%%%%%%%%%%%%%%%%%%%%%%%%%%%%%%%%%%%%%%%%%%%%%%%%%%%%%%%%%%%%%%%%%%%%%%%%%%%%%%%%%%%%%%%%%%%
\newcommand{\dem}{\noindent\textbf{\underline{Prova}.}\hspace{0.6cm}}   %começo de uma demonstracao
\newcommand{\cqd}{$\hfill\blacksquare$}                                  %fim de uma demonstração
\newcommand{\Q}{\mathbb{Q}}
\newcommand{\C}{\mathbb{C}}
\newcommand{\Z}{\mathbb{Z}}
\newcommand{\R}{\mathbb{R}}
\newcommand{\N}{\mathbb{N}}
%\newcommand{\C}{\mathbb{Cr}}
\newcommand{\supp}{\textrm{supp}~}
\newcommand{\Aut}{{\rm Aut}}
\renewcommand{\b}{{\partial}}
\def\join{\ast}
%========================================================================================================
 \DeclareMathOperator*{\esup}{\textrm{ess\,sup}~}
 \DeclareMathOperator*{\re}{Re\,}
 \DeclareMathOperator*{\im}{Im\,}
 \DeclareMathOperator*{\sen}{sen}
 \DeclareMathOperator*{\spa}{span}
 \DeclareMathOperator*{\tg}{tg}
 \DeclareMathOperator*{\ctg}{ctg}
 \DeclareMathOperator*{\grau}{grau}
 \DeclareMathOperator*{\diag}{diag}
 \DeclareMathOperator*{\trac}{tra\c{c}o}
%=======================================================================================================
\newcommand{\pr}{\hspace{0.67cm}}
\def\noi{\noindent}
\def\dis{\displaystyle}



%%%%%%%%%%%%%%%%%%%%%%%%%%%%%%%%%%%%%%%%%%%%%%%%%%%%%%%%%%%%%%%%%%%%%%%%%%%%%%%%%%%%%%%%%%%%%%%%%
%......................Defini\c{c}\~{a}o de Teoremas, Corol\'{a}rios, etc........................
%%%%%%%%%%%%%%%%%%%%%%%%%%%%%%%%%%%%%%%%%%%%%%%%%%%%%%%%%%%%%%%%%%%%%%%%%%%%%%%%%%%%%%%%%%%%%%%%%
\newtheorem{teo}{Teorema}[section]
\newtheorem{prop}{Proposi\c{c}\~{a}o}[section]
\newtheorem{definicao}{Defini\c{c}\~{a}o}[section]
\newtheorem{lema}{Lema}[section]
\newtheorem{corolario}{Corol\'{a}rio}[section]
\newtheorem{obs}{Observação} [section]        %
\newtheorem{exemplo}{Exemplo}[section]     %
\newtheorem{problema}{Problema}[section]



\setlength{\parindent}{4ex} \setlength{\parskip}{2ex}
\usepackage{beamerthemeshadow}

\vspace{-7cm}


\title{\textbf{Transformação linear aplicada em sistemas de cores}} %colocar titulo

\vspace*{-0.5cm}
\author[Pietropaolo D. C.; Galves, A. P. T.]{
  Domitila Crispim Pietropaolo - Bolsista PICME \\
  Orientadora: Prof. Dra. Ana Paula Tremura Galves } %colocar seu nome e da orientadora

\vspace*{-0.5cm}

\institute[Universidade Federal de Uberl\^{a}ndia]{
    Faculdade de Matem\'{a}tica \\
    Universidade Federal de Uberl\^{a}ndia}

  \vspace*{-0.65cm}

\date{VI Mostra de Inicia\c{c}\~{a}o Cient\'{i}fica \\ Maio de 2017} %mudar o nome do evento


\begin{document}


\frame{\titlepage}
\part<presentation>{Main Talk} % tira nomes de se\c{c}\~{o}es escritos no cabe\c{c}alho da p\'{a}gina inicial


\AtBeginSection[]{
  \frame<handout:0>{
    \frametitle{Sum\'{a}rio}
    \tableofcontents[current,currentsection]
  }
} %comando que gera o sumario

%:::::::::::::::::::::::::::::::::::::::::::::::::::::::::::::::::::::::::::::::::::::::::::::::::::::::::::::::::::::::::::::::::::::::::::::::::::::::::::
%todo slide \'{e} definido comecando com \frame{ e terminando com }

\section{Objetivo}

\frame{\frametitle{Objetivo} %colocar aqui dentro o nome da secao para aparecer escrito no slide de forma destacada

A ideia central do trabalho é fazer a conversão entre sistemas de cores (XYZ e RGB, CMY e RGB, YIQ e RGB) utilizando conceitos simples de álgebra linear, tais como transformação linear e mudança de base no espaço vetorial.

% colocar o objetivo

 }

%:::::::::::::::::::::::::::::::::::::::::::::::::::::::::::::::::::::::::::::::::::::::::::::::::::::::::::::::::::::::::::::::::::::::::::::::::::::

\section{Metodologia}

\subsection{Fundamentação teórica}
\frame{\frametitle{Transformação linear}

\begin{definicao}
 Seja $T: U \to V$ uma transformação linear com $U$ e $V$ espaços vetoriais de dimensão finita. Sejam $B=\{u_1, ..., u_n\}$ uma base de $U$ e $C=\{v_1, ..., v_m\}$ uma base de $V$, podemos escrever $T(u_j)= a_{1j}v_1 +...+ a_{mj}v_m$, $j=1,..., n$. A matriz:
 $$\left(
     \begin{array}{cccc}
       a_{11} & a_{12} & \cdots & a_{1n} \\
       a_{21} & a_{22} & \cdots & a_{2n} \\
       \vdots & \vdots & \ddots & \vdots \\
       a_{m1} & a_{m2} & \cdots & a_{mn} \\
     \end{array}
   \right) \in M_{m\times n}
 $$
\noindent é chamada matriz de transformação $T$ com relação as bases $B$ e $C$.
\end{definicao}
 }

%:::::::::::::::::::::::::::::::::::::::::::::::::::::::::::::::::::::::::::::::::::::::::::::::::::::::::::::::::::::::::::



%::::::::::::::::::::::::::::::::::::::::::::::::::::::::::::::::::::::::::::::::::::::::::::::::::::::::::::::::::::::::::::::::::::::


\section{Resultados}

\frame{
    \frametitle{Sistema RGB}

O olho humano possui três tipos de células cones, as quais captam as cores e levam-nas até o cérebro. Young-Helmholtz estabelece, no seu modelo tricromático, que sistemas de processamento de cor do olho humano baseiam-se na amostragem das faixas vermelha, verde e azul do espectro visível, feitas pelas moléculas fotossensíveis do olho. E com isso, surge o primeiro modelo padrão básico: CIE-RGB (CIE: Commission Internationale de l'Eclairage).
}

\frame{
   \centerline{\includegraphics[scale=0.15]{rgb.jpg}}
     \vspace{0.3cm}
     \centerline{Figura ilustrando o Sistema RGB}
}

\frame{
\frametitle{O modelo XYZ e a mudança de coordenadas em relação ao RGB}
Devido a incapacidade que o monitor possuía em projetar todas as cores visíveis pelo olho humano, o sistema CIE-XYZ foi criado para trazer modificações de proporções de intensidade das cores.}
\frame{
No sistema XYZ as coordenadas das cores primárias (RGB) são dadas pelos seguintes vetores: $R = (0.73467, 0.26533, 0.0)$, $G = (0.27376, 0.71741, 0.00883)$, $B = (0.16658, 0.00886, 0.82456)$, os quais correspondem aos vetores $R=(1,0,0)$, $G=(0,1,0)$ e $B=(0,0,1)$ no sistema RGB. Para obter uma transformação entre esses dois sistemas devemos buscar um outro conjunto de vetores comum aos espaços XYZ e RGB.
}

\frame{
Tome os vetores que correspondem a cor branca em cada um dos sistemas, que tem coordenadas $\displaystyle \left(\frac{1}{3}, \frac{1}{3}, \frac{1}{3}\right)$. Dessa forma, a transformação linear do sistema RGB no sistema XYZ é definida por $T(R,G,B)=(X,Y,Z)$ e, usando o fato que $T\displaystyle \left(\frac{1}{3}, \frac{1}{3}, \frac{1}{3}\right)=\displaystyle \left(\frac{1}{3}, \frac{1}{3}, \frac{1}{3}\right)$ obtemos uma matriz de transformação de coordenadas dada por:
$$ \left(
\begin{array}{ccc}
 X \\
Y\\
Z
\end{array}
\right) =
 \left(
\begin{array}{ccc}
0.49   & 0.17697    & 0  \\
0.31  & 0.81240     & 0.01 \\
0.2   & 0.01063 & 0.99
\end{array}
\right)
.
\left(
\begin{array}{ccc}
 R  \\
G \\
B
\end{array}
\right)
$$

\noindent que faz a mudança de coordenadas no sistema RGB para o sistema XYZ.
 }
\frame{
\centerline{\includegraphics[scale=0.5]{xyz.jpg}}
\centerline{Figura ilustrativa de como XYZ se representa no plano}
}
\frame{
Tentando obter coordenadas RGB a partir de um vetor CMY V=(-0.0210; 0.6121; 0.4876).

    $$ \left(
\begin{array}{ccc}
 X \\
Y\\
Z
\end{array}
\right) =
 \left(
\begin{array}{ccc}
0.49   & 0.17697    & 0  \\
0.31  & 0.81240     & 0.01 \\
0.2   & 0.01063 & 0.99
\end{array}
\right)
.
\left(
\begin{array}{ccc}
 -0.021 \\
0.6121 \\
0.4876
\end{array}
\right)
$$

Fazendo as contas temos que  V = (0.0980, 0.4956, 0.4850) é o vetor que procuramos .
}
 \frame {
    \frametitle{O modelo CMY e a mudança de coordenadas em relação ao RGB}

Baseando-se nas cores complementares, o CMY é designado por modelo subtrativo de cor, opondo-se ao modelo RGB que é designado por modelo aditivo de cor. É muito utilizado em impressão a cores em papel branco (como já foi citado na introdução).

A transformação do espaço CMY no espaço RGB é a mais simples, basicamente consiste em
$$
\left(
\begin{array}{ccc}
R    \\
G  \\
B
\end{array}
\right)
=\left(
\begin{array}{ccc}
1    \\
1  \\
1
\end{array}
\right)
-
\left(
\begin{array}{ccc}
C    \\
M  \\
Y	
\end{array}
\right).
$$
}
\frame{
   \centerline{\includegraphics[scale=0.6]{cubo.jpg}}
     \vspace{0.3cm}
     \centerline{Figura ilustrando o Sistema CMY}
}
\frame{
Tentando obter coordenadas RGB a partir de um vetor CMY V=(0.212, 0.565, 0.354).

 $$
\left(
\begin{array}{ccc}
R    \\
G  \\
B
\end{array}
\right)
=\left(
\begin{array}{ccc}
1    \\
1  \\
1
\end{array}
\right)
-
\left(
\begin{array}{ccc}
0.212   \\
 0.565 \\
0.354
\end{array}
\right).
$$

Fazendo as contas temos que  V =(0.788, 0.435, 0.646) é o vetor que procuramos .
}
\frame{
\frametitle{O modelo YIQ e a mudança de coordenadas em relação ao RGB}
Utilizado no sistema NTSC (National Television Standards Committee) o modelo YIQ foi criado para permitir que as emissões dos sistemas de televisão em cores fossem compatíveis com os receptores em preto e branco.

 \vspace{0.3cm}

\centerline{\includegraphics[scale=0.25]{televisao.jpg}}

}
\frame{ Baseia-se na divisão dos sinais de cor RGB em um sinal de luminosidade, ou luminância $(Y)$, dada por:
 $$ Y=0.299\, R+ 0.587 \, G+ 0.114 \, B. $$

Os parâmetros $I$ e $Q$ estão relacionados às cores propriamente, então envolvem luminância e as cores RGB em suas fórmulas, como podemos observar:
$$I=0.74(R-Y)-0.27(B-Y)$$
$$Q=0.48(R-Y)+0.41(B-Y).$$
}

\frame{
Dessa forma, isolando as variáveis $Y$, $I$ e $Q$ em função de $R$, $G$ e $B$, chegamos na seguinte matriz de transformação do espaço de cor RGB para o espaço YIQ
$$ \left(
\begin{array}{ccc}
 Y  \\
I \\
Q
\end{array}
\right) =
 \left(
\begin{array}{ccc}
0.299   & 0.587    & 0.114  \\
0.596  & -0.275      & -0.321 \\
0.212   & -0.523 & -0.311
\end{array}
\right)
.
\left(
\begin{array}{ccc}
 R  \\
G \\
B
\end{array}
\right)
$$
}

\frame{
Tentando obter coordenadas YIQ a partir de um vetor RGB V=(0.213, 0.772 , 0.564).

$$ \left(
\begin{array}{ccc}
 Y  \\
I \\
Q
\end{array}
\right) =
 \left(
\begin{array}{ccc}
0.299   & 0.587    & 0.114  \\
0.596  & -0.275      & -0.321 \\
0.212   & -0.523 & -0.311
\end{array}
\right)
.
\left(
\begin{array}{ccc}
0.213  \\
0.772 \\
0.564
\end{array}
\right)
$$

Fazendo as contas temos que  V = (0.5808, -0.2670, -0.5342) é o vetor que procuramos .

}
%::::::::::::::::::::::::::::::::::::::::::::::::::::::::::::::::::::::::::::::::::::::::::::::::::::::::::::::::::::::::::::::::::::::::::::::::
\section{Conclus\~{a}o}

\frame{
No trabalho podemos observar que a álgebra linear auxiliou processos de cálculo para que as cores possam ser exibidas e transmitidas em diferentes plataformas, como televisores, monitores, etc. O ponto mais importante desse trabalho é como vários sistemas de cores podem ser reduzidos a espaços vetoriais e manipulados facilmente através de transformações lineares.
}

%::::::::::::::::::::::::::::::::::::::::::::::::::::::::::::::::::::::::::::::::::::::::::::::::::::::::::::::::::::::::::::::::::::::::::::::::::::

\frame{\frametitle{Refer\^{e}ncias bibliogr\'{a}ficas}
\begin{thebibliography}{9}

{\beamertemplatearticlebibitems
\bibitem{zill} Biezuner, R. J.; Macedo, E. A. A.; Moreira, B. T. \textit{Mudanças de coordenadas em sistemas de cores}. Belo Horizonte: UFMG. Google Acadêmico. PDF.}
{\beamertemplatearticlebibitems
\bibitem{mimi} Gomes, J. (1994). \textit{Computação gráfica: imagem}. Rio de Janeiro: IMPA/SBM.}

{\beamertemplatebookbibitems
\bibitem{barreto} Howard, A.; Rorres, C. (2001). \textit{Álgebra linear com aplicações}. Porto Alegre: Bookman.} % colocar essa referencia quando for artigo

\end{thebibliography}
}


% estilo para aparecer um icone de um livro
    %\beamertemplatebookbibitems
    % estilo para aparecer um icone de um paper =)
    %\beamertemplatearticlebibitems


\frame{
\centerline{\textbf{\underline{\Huge{Obrigado!  }}}}}

\end{document} 
